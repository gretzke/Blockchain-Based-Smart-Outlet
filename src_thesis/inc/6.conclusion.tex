\section{Conclusion}
The goal of this thesis was to examine whether M2M payments could be implemented on a microcontroller level. A prototype of an electrical plug and socket were developed that could transact monetary value and electricity with barely any human interaction. The transactions were conducted through a Smart Contract running on the Ethereum blockchain, securing both parties from fraud. Not only were all requirements, set in the beginning of this thesis, met satisfactorily, but also exceeded expectations. A payment channel was successfully implemented, allowing both microcontrollers to send and receive transactions with near instant transaction times and almost no fees, which should average at less than 0.10\euro{} per electricity sale at the time of writing.
\\\\
Comparing payment channel transactions to traditional payments in terms of speed, it should be relatively equal. Off-chain transactions are probably a bit faster in transmission, as they only need to be transferred locally and no communication with a server is required. The clear advantage of the payment channel is the transaction price and the modularity of the transactions. No matter how many transactions are sent, only the first and the last are written to the blockchain, enabling the transmission of thousands of transactions at just a fraction of the cost of traditional payments.
\\\\
However the benefits of the payment channel come at a cost. Actually not just one, but three costs. The first cost is the computational cost. Every off-chain transaction requires the generation of a transaction on the side of the customer and the verification of said transaction on the side of the supplier. Having to do this computational work a few times a minute adds up. Additionally a lot of memory is required when handling Smart Contract requests. 
\\
The second cost is the security cost. A huge advantage and disadvantage of Smart Contracts is their immutability. If the Smart Contract code can’t be changed, a party cannot be stolen from through updating the Smart Contract with new code. On the other hand a bug cannot be fixed once the Smart Contract has been deployed to the blockchain. Unfortunately this already resulted in losses of multiple hundreds of millions of dollars on the Ethereum blockchain alone. Even every part of the microcontroller has to be carefully secured, so under no circumstances the private key can be extracted to steal funds.
\\
The third cost is the economic cost. Cryptocurrencies and blockchain technology is still in its infancy and therefore the price is extremely volatile. A sudden rise or drop of 40\% in price is not rare and would heavily influence the customer experience. Additionally the adoption is very low, only a few people own cryptocurrencies and even less actually use it in day to day life. Requiring payments in cryptocurrency would not be economically viable.
\\\\
As cryptocurrencies are still in its infancies, most of the development is still ahead and nobody can imagine what they will look like in 10 years, similarly when the internet was in its infancy, no-one even imagined that something like Facebook could even exist. The technology will scale to process more transactions at even lower costs and will be steadily integrated more in more into our daily lives. For example in its latest flagship, the Galaxy S10, Samsung included an Ethereum and Bitcoin wallet, meaning that millions of new people are able to transact with cryptocurrencies securely without any big entry barriers.
\\\\
Eventually the price of cryptocurrencies will stabilize, but until that happens, some suggestions for future work that can be build on this thesis are suggested. Currently the microcontrollers are communicating with each other over WebSockets and the plug has to be powered separately to function. An implementation that would improve upon this, would have the socket and the plug communicate over \abbr{power-line communication}{PLC}, i.e. over the electricity that is transmitted. Before the actual electricity delivery, the socket could limit the output so that the microcontroller of the plug can be powered and communicated with, without needing an external energy source. Another interesting topic would be to implement the payment system with other cryptocurrencies, such as IOTA or Nano to analyze the key strengths and weaknesses of the different payment methods. As fast as the crypto-ecosystem develops, there could be new cryptocurrencies in a few months that couldn’t even be considered during this thesis. In about a year at the time of writing, Ethereum plans to upgrade to a new blockchain, also called Ethereum 2.0\cite{eth-2} which will implement a new mining algorithm and lots of different scaling solutions, so who knows what the future has in store?
