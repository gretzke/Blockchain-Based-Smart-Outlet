\section{Appendix}
\label{sec:appendix}

% \begin{listing}[H]
% \begin{minted}[linenos, bgcolor=light-gray, fontsize=\scriptsize]{cpp}
% pragma solidity ^0.5.0;

% /**
%  * @title SafeMath
%  * @dev Unsigned math operations with safety checks that revert on error
%  * @notice https://github.com/OpenZeppelin/openzeppelin-solidity
%  */
% library SafeMath {
%     /**
%      * @dev Multiplies two unsigned integers, reverts on overflow.
%      */
%     function mul(uint256 a, uint256 b) internal pure returns (uint256) {
%         // Gas optimization: this is cheaper than requiring 'a' not being zero, but the
%         // benefit is lost if 'b' is also tested.
%         // See: https://github.com/OpenZeppelin/openzeppelin-solidity/pull/522
%         if (a == 0) {
%             return 0;
%         }

%         uint256 c = a * b;
%         require(c / a == b);

%         return c;
%     }

%     /**
%      * @dev Integer division of two unsigned integers truncating the quotient, reverts on division by zero.
%      */
%     function div(uint256 a, uint256 b) internal pure returns (uint256) {
%         // Solidity only automatically asserts when dividing by 0
%         require(b > 0);
%         uint256 c = a / b;
%         // assert(a == b * c + a % b); // There is no case in which this doesn't hold

%         return c;
%     }

%     /**
%      * @dev Subtracts two unsigned integers, reverts on overflow (i.e. if subtrahend is greater than minuend).
%      */
%     function sub(uint256 a, uint256 b) internal pure returns (uint256) {
%         require(b <= a);
%         uint256 c = a - b;

%         return c;
%     }

%     /**
%      * @dev Adds two unsigned integers, reverts on overflow.
%      */
%     function add(uint256 a, uint256 b) internal pure returns (uint256) {
%         uint256 c = a + b;
%         require(c >= a);

%         return c;
%     }

%     /**
%      * @dev Divides two unsigned integers and returns the remainder (unsigned integer modulo),
%      * reverts when dividing by zero.
%      */
%     function mod(uint256 a, uint256 b) internal pure returns (uint256) {
%         require(b != 0);
%         return a % b;
%     }
% }

% contract SocketPaymentChannel {
%     using SafeMath for uint256;

%     address public owner;
%     uint256 public pricePerSecond;
%     mapping(address => uint256) public balances;
%     // timeout in case no one closes the payment channel in seconds
%     uint256 public expirationDuration;
%     uint256 public minDeposit;

%     // global payment channel variables
%     bool public channelActive;
%     uint256 public creationTimeStamp;
%     uint256 public expirationDate;
%     address public channelCustomer;
%     uint256 public maxValue;

%     // use nonces to avoid replay attacks
%     mapping(address => uint256) public customerNonces;

%     event InitializedPaymentChannel(address indexed customer, uint256 indexed start, uint256 indexed maxValue ,uint256 end);
%     event ClosedPaymentChannel(address indexed sender, uint256 indexed value, bool indexed expired, uint256 duration);
%     event PriceChanged(uint256 indexed oldPrice, uint256 indexed newPrice);
%     event Withdrawal(address indexed sender, uint256 indexed amount);

%     modifier onlyOwner() {
%         require(msg.sender == owner, "sender is not owner");
%         _;
%     }

%     constructor(uint256 _pricePerSecond, uint256 _expirationDuration, uint256 _minDeposit) public {
%         owner = msg.sender;
%         pricePerSecond = _pricePerSecond;
%         expirationDuration = _expirationDuration;
%         minDeposit = _minDeposit;
%         channelActive = false;
%     }

%     function initializePaymentChannel() public payable returns (bool) {
%         require(!channelActive, "payment channel already active");
%         require(msg.value >= minDeposit, "minimum deposit value not reached");

%         // set global payment channel information
%         channelActive = true;
%         channelCustomer = msg.sender;
%         maxValue = msg.value;
%         creationTimeStamp = now;
%         expirationDate = now.add(expirationDuration);

%         // It's cheaper to use msg.sender instead of channelCustomer, msg.value instead of maxValue, etc.
%         emit InitializedPaymentChannel(msg.sender, now, now.add(expirationDuration), msg.value);
%         return true;
%     }

%     function closeChannel(bytes memory _signature, uint256 _value, uint256 _nonce) public onlyOwner returns (bool) {
%         require(channelActive, "payment channel not active");
%         // TODO check nonce before payout
%         require(verifySignature(_signature, _value, _nonce), "signature not valid");

%         // increase nonce after payment channel is closed
%         customerNonces[channelCustomer] = customerNonces[channelCustomer].add(1);

%         // if maxValue is exceeded, set value to maxValue
%         if (_value > maxValue) {
%             _value = maxValue;
%             balances[owner] = balances[owner].add(_value);
%         } else {
%             balances[owner] = balances[owner].add(_value);
%             balances[channelCustomer] = balances[channelCustomer].add(maxValue.sub(_value))
%         }

%         emit ClosedPaymentChannel(msg.sender, _value, false, now - creationTimeStamp);

%         channelActive = false;
%         channelCustomer = 0;
%         maxValue = 0;
%         expirationDate = 0;
%         creationTimeStamp = 0;

%         return true;
%     }

%     function timeOutChannel() public returns (bool) {
%         require(channelActive, "payment channel not active");
%         require(now > expirationDate, "payment channel not expired yet");

%         // increase nonce after payment channel is closed
%         customerNonces[channelCustomer] = customerNonces[channelCustomer].add(1);

%         // return funds to customer if channel was not closed before channel expiration date
%         balances[channelCustomer] = balances[channelCustomer].add(maxValue);

%         emit ClosedPaymentChannel(msg.sender, 0, true, now - creationTimeStamp);

%         channelActive = false;
%         channelCustomer = 0;
%         maxValue = 0;
%         expirationDate = 0;
%         creationTimeStamp = 0;

%         return true;
%     }

%     /** 
%      */
%     function verifySignature(bytes memory _signature, uint256 _value, uint256 _nonce) public view returns (bool) {

%         // split signature into r,s,v values (https://programtheblockchain.com/posts/2018/02/17/signing-and-verifying-messages-in-ethereum/)
%         require(_signature.length == 65, "signature length is not 65 bytes");

%         bytes32 r;
%         bytes32 s;
%         uint8 v;

%         // split using inline assembly
%         assembly {
%             // first 32 bytes of message
%             r := mload(add(_signature, 32))
%             // second 32 bytes of message
%             s := mload(add(_signature, 64))
%             // first byte of the next 32 bytes
%             v := byte(0, mload(add(_signature, 96)))
%         }

%         // variables that are included in the message: value, address of contract, nonce
%         address contractAddress = address(this);
%         // hash variables
%         bytes32 message = keccak256(abi.encodePacked(_value, contractAddress));
%         // prefix message with ethereum specific 
%         bytes32 prefixedMessage = keccak256(abi.encodePacked("\x19Ethereum Signed Message:\n32", message));
%         // returns true if recovered address is equal to customer address
%         return ecrecover(prefixedMessage, v, r, s) == channelCustomer;
%     }

%     function withdraw() public returns (bool) {
%         uint256 balance = balances[msg.sender];
%         balances[msg.sender] = 0;
%         msg.sender.transfer(balance);
%         emit Withdrawal(msg.sender, balance);
%         return true;
%     }

%     function changePrice(uint256 _newPrice) public onlyOwner {
%         emit PriceChanged(pricePerSecond, _newPrice);
%         pricePerSecond = _newPrice;
%     }
% }
% \end{minted}
% \caption{Smart Contract code}
% \label{lis:sc}
% \end{listing}

