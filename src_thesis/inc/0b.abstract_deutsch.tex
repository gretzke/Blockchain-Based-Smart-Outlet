\thispagestyle{empty}

\section*{Abstract}

Für Stadtbewohner ist es häufig schwierig, direkt vor dem eigenen Haus einen Parkplatz zu finden.
In Zukunft, wenn immer mehr Elektroautos auf den Straßen fahren, kann das Laden des Autos zu einem Problem werden.
Diese Bachelorarbeit entwickelt ein Peer-to-Peer Konzept, mit dem jeder Hausbesitzer über eine intelligente Steckdose Strom an Elektroautobesitzer ohne Mittelsmann verkaufen kann.
Darüber hinaus wurde ein Zahlungssystem mit Hilfe einer auf der Blockchain-Technologie basierenden, dezentralen Rechenplattform entwickelt, welches auf eingebetteten Plattformen, wie einer intelligenten Steckdose und einem intelligentem Stecker, läuft.
Dieses Konzept und das daraus hervorgehende Zahlungssystem wurden erfolgreich implementiert und erlauben den sicheren Austausch von Elektrizität gegen einen finanziellen Geldwert auf eingebetteten Plattformen.