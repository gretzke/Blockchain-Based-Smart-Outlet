\thispagestyle{empty}

\section*{Abstract}

Elektroautos sind mittlerweile ein wichtiger Bestandteil der Automobilindustrie.
Der Mangel einer ausgebauten Infrastruktur verfügbarer Ladestationen hemmt Autofahrer häufig vor dem Wechsel zu einem Elektroauto.
Für Stadtbewohner ist es häufig schwierig, direkt vor dem eigenen Haus einen Parkplatz zu finden, was das Laden des Autos über Nacht erschwert.
Um eine Lösung für dieses Problem zu präsentieren, wurde ein Peer-to-Peer Konzept entwickelt, mit dem jeder Hausbesitzer über eine intelligente Steckdose Strom an Elektroautobesitzer verkaufen kann.
Des Weiteren wurde ein auf der Blockchain-Technologie basierendes Zahlungssystem entwickelt, welches Machine-to-Machine Zahlungen auf eingebetteten Systemen wie einem intelligenten Stecker oder einer Steckdose ermöglicht.
Dabei fungiert ein Smart Contract auf der Ethereum Blockchain als Mittelsmann, der die Sicherheit der Transaktion gewährleistet.
Somit werden keine externen dritte Parteien im Zahlungsprozess benötigt.
Dieses Konzept und Zahlungssystem wurden erfolgreich auf eingebetteten Plattformen implementiert und erlauben den sicheren Austausch von Elektrizität gegen einen finanziellen Geldwert.