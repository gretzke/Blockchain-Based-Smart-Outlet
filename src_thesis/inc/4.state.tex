\section{State of the art}

A few companies already started researching and experimenting with blockchain technologies and even went as far as combining it with e-mobility. A company called accessec GmbH built a prototype of a car wallet that works with the IOTA cryptocurrency and integrates a point of sale allowing to conduct transactions with vendors seamlessly\cite{car-wallet}. Bosch teamed up with energy supplier EnBW to build a prototype a blockchain based charging station\cite{bosch-dlt}. The project that came closest to the topic of this bachelor’s thesis is called Share\&Charge\cite{share-charge}, formerly known as Blockcharge\cite{blockcharge}, which was founded by Innogy, a subsidiary of the energy company RWE\cite{innogy}. It launched at the end of April 2017 with close to 1,500 charging stations, but the project was closed merely a year later\cite{share-charge-closed}. It claimed to be a P2P charging network calling itself the "AirBnB of Charging Stations". It was running on the Ethereum Mainnet where anyone could become a charging station owner by purchasing a smart electrical socket to sell electricity. At first glance it seemed like the solution this bachelor’s thesis was trying to achieve, but upon further investigation the decentralization aspect had to be questioned. The electrical socket was communicating with a smartphone app to manage the purchase of electricity. This app had to be preloaded with fiat currency via PayPal, etc., the transaction was conducted in fiat and a charging station owner could only withdraw fiat currency as well. There is no evidence that the monetary transaction itself was handled on the blockchain and not just the record aspect of it. Additionally it was claimed that the system worked without a middleman but the fee that had to be paid to Share\&Charge with every charging process contradicts that claim. All this information led to the conclusion that the private keys probably were not handled by the users themselves which is a crucial point in building decentralized applications.
\\
To summarize, all projects combining blockchain with e-mobility seem to be bringing existing centralized business models to the blockchain and could work just as well with traditional payment methods. Furthermore no technical information, let alone source code, could be found that this bachelor’s thesis could use or improve upon.
\\\\
Looking at the implementation of the interaction with the Ethereum network on a microcontroller level it was discovered that it is very far from production ready, as it is on high level programming languages. Although utility libraries exist, many of them were not compatible with the microcontroller and the ones that were, often had to have bugs fixed or additional functions implemented. Additionally, Smart Contracts mainly work with 256 bit unsigned integers for numbers which have to be passed as hexadecimal strings, which makes running the necessary code on hardware with very limited memory all the more challenging. All in all, Ethereum development on microcontrollers has a long way to go before it becomes easy to implement prototypes on it.
\\\\
Lastly, payment channels were implemented in this bachelor’s thesis. They are a subset of state channels, which currently are a heavily researched topic on their own\cite{state-channels}. As far as research went, every state channel implementation was still under development and not a single implementation of a payment channel on a microcontroller level could be found.