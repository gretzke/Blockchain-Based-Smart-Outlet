\section{Introduction}
\begin{quote}
  Whereas most technologies tend to automate workers on the periphery doing menial tasks, blockchains automate away the center. Instead of putting the taxi driver out of a job, blockchain puts Uber out of a job and lets the taxi drivers work with the customer directly.
  \\
  {\textit{— Vitalik Buterin, co-founder of Ethereum}}
\end{quote}

A cryptocurrency based on a blockchain was first implemented in 2009 by Satoshi Nakamoto (pseudonym) and was called Bitcoin.
Since then, it has steadily gained importance every year.
Meanwhile, thousands of cryptocurrencies and tokens were built on this technology.
The hype in the year 2017 called the attention of many companies to blockchain and even last year, when the value of cryptocurrencies fell as far as 95\%, the interest in this field did not drop.
\\\\
Compared to traditional payment methods like Visa, Banks and PayPal, cryptocurrencies are built decentralized, meaning that there is no central organization that controls transactions, the issuance of new money, et cetera.
The validity of the blockchain \abbr{peer to peer}{P2P} network is secured through cryptographic protocols.
This brings several benefits.
Traditional payment methods usually go with high transactions costs, most commonly in the amount of a few percent.
On the contrary, the cost of a single transaction on a blockchain averages out at just a few cents\cite{ethereum-fee}.
Some cryptocurrencies even work without any fees.
\\\\
Because of this, they are suited for micro transactions really well.
There are some disadvantages, though.
The blockchain technology is still at an early stage and really immature.
Compared to traditional electronic payments, it only manages to achieve very few \abbr{transactions per second}{TPS} and has long transaction times.
E.g., Bitcoin manages 4-5 TPS\cite{bitcoinTPS} as opposed to Visa, which manages to process almost 4,000 TPS on average\cite{visa}.
\\\\
As stated in the quote above, the key strength of blockchain and cryptocurrencies is the decentralization aspect.
For many, it will reshape various markets we know today, potentially revolutionize the financial industry and even disrupt monopolies in the future.
\\\\
Another trend regarding the future are electric cars.
It's expected that in a few years most cars on the road and almost all cars sold will be electric.
Often these need to be charged overnight.
Unfortunately, most city residents are familiar with the problem that they rarely park in front of their own house, let alone own a garage.
It's foreseeable that recharging a car might bring difficulties.
\\\\ 
This bachelor thesis is devoted to this problem.
It examines whether a smart electrical socket, which is placed outside the house by a homeowner, can be used to efficiently sell electricity and which payment method is suited best for this task.
Based on an initial concept, a prototype is to be developed that implements the previously worked out features.
It will serve as an example on how to implement \abbr{machine to machine}{M2M} payments on a microcontroller level.