% !TEX root = ../diss.tex

\thispagestyle{empty}


\begin{center}
  \textbf{Acknowledgments}
\end{center}

%\todo{Pending}.


My research was carried out between 2014 and 2018 in the Microelectronic Systems Design Research Group, and as part of the Research Training Group RTG 1932 ``\textit{Stochastic Models for Innovations in the Engineering Sciences}'', 
% (Project P2 ``Stochastic models for system-on-chip design and Monte Carlo hardware acceleration'') 
at the University of Kaiserslautern, Germany. 
I gratefully acknowledge financial support from the Deutsche Forschungsgemeinschaft (DFG) within the RTG 1932.
Furthermore,  I would like to express special gratitude to those who have had a direct positive influence in my research and in this thesis:
%I would like to express a special gratitute to:
  
%I would like to thank all my colleagues (from the Microelectronic Systems Design Research Group, from the Research Trainning Group RTG 1932, and from the Department of Mathematics) with whom I have had the chance to interact and from whom I have had the opportunity to gain invaluable knowledge. 

%In particular, I would like to express special gratitude to those who had a direct positive influence in this thesis:

%So far, I have found the interdisciplinary research between mathematics and engineering im-mensely profitable, very interesting and certainly fun. 
%Since my background is in engineering, it has been a continuous, yet rewarding, challenge to work alongside mathematicians, but at the same time a great chance to consider the problems at hand from a different perspective. 
%This is true not only in the scope of the P2 project where I am involved, but also for the entire group. In this regard, the off-campus workshops, as well as the summer schools, have been a great opportunity to share experiences and ideas with other colleagues.
%

\begin{itemize}
	\vspace{-2mm}
	\item Prof. Dr.-Ing- Norbert Wehn, for providing me with the opportunity of pursuing my PhD in his research group, for his guidance, for continuously raising the expectation bar, and for the support in attending the conferences and workshops that allowed me to advance in my research;

	\vspace{-2mm}
	\item Prof. Dr. Ralf Korn, for the opportunity of joining the RTG 1932, for his guidance, for the useful discussion regarding finance, and for the support in attending the workshops from which I could very much profit, both personally and professionally;

	\vspace{-2mm}
	\item Dr. Sascha Desmettre, for the know-how and theoretical support in the field of risk management, and for all the hard work and cooperation which yield two conference papers and one extensive article in the prestigious Risks journal;
%, and for the discussions we have hold in this field.

	\vspace{-2mm}
	\item Dr. Sema Coskun, my mathematics counterpart in the P2 project (RTG 1932) with whom I worked before and after her graduation, for the great cooperation atmosphere, and for the useful discussions on option pricing and stochastic models;

	\vspace{-2mm}
	\item Dr. Songyin Tang, both before and after his graduation, for all the support, for his willingness and readiness to cooperate, and for all the hard work that yield three conference papers and one book chapter.
In this regard, I would also like to thank Dr. Qian Liang, with whom I had the chance to work in one of those papers;

	\vspace{-2mm}
	\item Dr. Steffen Omland, before his graduation, for suggesting the Brownian Bridge approach for the \acrlong{LS} algorithm, which yield a  conference paper;

	\vspace{-2mm}
	%\item Dr.-Ing. Christian Weis, for providing invaluable support and guidance whenever I needed it, and for critically reviewing the draft version of many of my papers;
	\item Dr.-Ing. Christian Weis, for the useful discussions, for his support with memory-related topics, and for critically reviewing the draft version of some of my papers;
	%\item Dr.-Ing. Christian Weis, for his support, and for critically reviewing the draft version of many of my papers;
	%\item Dr.-Ing. Christian Weis, for the useful discussions, both on memory-related topics and professionally, and for critically reviewing the draft version of many of my papers;

	\vspace{-2mm}
	\item Dipl.-Inf. Claus Kestel, for reviewing the grammar and coherence of the Zusammenfassung chapter in this thesis, which is written in German language;

	\vspace{-2mm}
	\item Martina Jahn, the secretary of our research group, for all the administrative support provided throughout my research in the group;
	\vspace{-2mm}
	\item Dipl.-Ing. (FH) Hans-Peter Goldhammer, Dipl.-Ing. (FH) Roland Volk, Andreas Christmann, and Markus Müller, for all their support in terms of infrastructure, hardware and software, and in particular during the setup of the high-performance workstation I later used in most part of my research.
\end{itemize}

I would also like to thank Michaela Blott and Kenneth O'Brien from Xilinx Inc., as well as Cathal McCabe from the Xilinx University Program, for all the support provided with the Xilinx SDAccel tool (which I have used during my research), and for the week I spent at Xilinx Research Labs in Dublin, in the year 2015, regarding this tool.

From the Microelectronic Systems Design Research Group I would also like to mention those with whom I had the chance to interact during my research:
Dr.-Ing. Imran Ali, 
Dr.-Ing. Christian Brugger, 
Dr.-Ing. Christian De Schryver, 
M.Sc. Muhammad Mohsin Ghaffar, 
Dr.-Ing. Matthias Jung, 
Dipl.-Ing. Matthias Herrmann, 
%Dipl.-Inf. Claus Kestel, 
M.Sc. Kira Kraft, 
M.Sc. Jan Lappas, 
Dipl.-Ing. Dominik Loroch, 
M.Sc. Deepak M. Mathew, 
M. Eng. Carl Rheinländer,
M.Sc. Vladimir Rybalkin, 
Dr.-Ing. Mohammadsadegh Sadri, 
Dipl.-Ing. Philipp Schläfer, 
Dr.-Ing. Stefan Scholl, 
M.Sc. Chirag Sudarshan,
M.Sc. Menbere Tekleyohannes, 
Dipl.-Math. Uwe Wasenmüller, 
%Dr.-Ing. Christian Weis, 
Dr.-Ing. Stefan Weithoffer, 
Dipl.-Ing. Sebastian Wille,
and Eng. Éder Ferreira Zulian.
%Without all their support this thesis could have never been written.

From the Research Training Group RTG 1932 ``Stochastic Models for Innovations in the Engineering Sciences'', and from the Department of Mathematics, I would also like to mention those with whom I had the chance to interact during my research:
%\todo{complete GrK List here, include Barbara and Julia}.
%\todo{ARPM (full)}.
%M.Sc. Florian Blandfort, 
M.Sc. Daniele Casucci, 
%Dr. Sema Coskun,
%Dr. Sascha Desmettre,
Dr. Jan Henrik Fitschen, 
Dr. Katharina Losch, 
Dr. Pak Hang Lo, 
%M.Sc. Kasem Maryamh, 
M.Sc. Lukas Mayer, % RECHECK
Dr. Anne Meurer, 
%(?) Adam Mühlbauer, % RECHECK
%(?) Nikita Nobel, % RECHECK
Dr. Merima Nurkanovic,
%Dr. Steffen Omland,
%Dipl.-Wirtsch.-Ing. Alexandra Pokhlestova, 
%M.Sc. Louisa Schlachter, 
Dipl.-Ing. Sebastian Schuff, 
Dr. Stefanie Schwaar, 
%Dr. Songyin Tang, 
M.Sc. Jingnan Wang, 
and Dipl.-Ing. Andreas Weber.
%as well as Dr. Qian Liang.


Finally, but not last, I want to thank my wife, Carolina, for her unconditional support and patience during the time I required for my research.
%This thesis is dedicated to my wife.
%Every minute invested in my PhD, is a minute that I had to borrow from her.
%


