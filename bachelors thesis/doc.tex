%%%%%%%%%%%%%%%%%%%%%%%%%%%%%%%%%%%%%
% Inlucudings:                      %
%%%%%%%%%%%%%%%%%%%%%%%%%%%%%%%%%{{{%
\documentclass[11pt,english,a4paper,chapterprefix]{scrartcl}
%\usepackage[T1]{fontenc}
\usepackage[bigcaptions]{listing}
%\usepackage[latin1]{inputenc}
\usepackage[small,bf,hang]{caption}
\usepackage[english]{babel}
%\usepackage{epsfig}
\usepackage{wrapfig}
%\usepackage{caption}
\usepackage{psfrag}
\usepackage[rflt]{floatflt}
\usepackage[usenames]{color}
\usepackage{graphicx}
\emergencystretch = 10pt
\usepackage{amsmath}
\usepackage{amssymb}
\usepackage{setspace}
%\usepackage{calc}
\usepackage{tocloft}
\usepackage{listing}
\usepackage{listings}
\usepackage{trsym}
\usepackage{trfsigns}
%\usepackage{minted}
\usepackage{multirow}
\usepackage{fancyhdr}
\usepackage{nomencl}
\usepackage{todonotes}
%\usepackage{float}
\usepackage{subfig}
\usepackage{url}
\usepackage{hyperref}
%\usepackage{listings}
%\input{subsections.sty}
\setcounter{secnumdepth}{5}
\setcounter{tocdepth}{5} 
\numberwithin{equation}{section}
\numberwithin{figure}{section}

%%%%%%%%%%%%%%%%%%%%%%%%%%%%%%%%%}}}%
% New Commands and Configurations:  %
%%%%%%%%%%%%%%%%%%%%%%%%%%%%%%%%%{{{%
%\setkomafont{section}{\Large\rmfamily}
%\setkomafont{subsection}{\large\rmfamily}
%\setkomafont{subsubsection}{\normalsize\rmfamily}
\setkomafont{paragraph}{\footnotesize}
\numberwithin{table}{section}
\numberwithin{listing}{section}
\setlength\textheight{24cm}
\definecolor{orange}{rgb}{1 , 0.5 , 0}
\definecolor{blue}{rgb}{0, 0 , 1}
\definecolor{green}{rgb}{0, 1 ,0}
\newcommand{\cb}{\textcolor{blue}}
\newcommand{\subsubsubsection}{\paragraph}
\newcommand{\subsubsubsubsection}{\subparagraph}
\clubpenalty = 10000
\widowpenalty = 10000
\displaywidowpenalty = 10000
\parindent0pt % No Indent
\makenomenclature
% Document Head
\begin{document}
%\restylefloat{figure}
\pagestyle{fancy}
\rhead{} 

\definecolor{light-gray}{gray}{0.90}

%\lstset{ %
%language=C,                % choose the language of the code
%basicstyle=\small\ttfamily
%,       % the size of the fonts that are used for the code
%%numbers=left,                   % where to put the line-numbers
%numberstyle=\footnotesize,      % the size of the fonts that are used for the line-numbers
%stepnumber=2,                   % the step between two line-numbers. If it's 1 each line 
%                                % will be numbered
%numbersep=5pt,                  % how far the line-numbers are from the code
%backgroundcolor=\color{light-gray},  % choose the background color. You must add \usepackage{color}
%showspaces=false,               % show spaces adding particular underscores
%showstringspaces=false,         % underline spaces within strings
%showtabs=false,                 % show tabs within strings adding particular underscores
%frame=single,                   % adds a frame around the code
%rulecolor= \color{light-gray},
%tabsize=2,                      % sets default tabsize to 2 spaces
%captionpos=b,                   % sets the caption-position to bottom
%breaklines=true,                % sets automatic line breaking
%breakatwhitespace=false,        % sets if automatic breaks should only happen at whitespace
%title=\lstname,                 % show the filename of files included with \lstinputlisting;
%                                % also try caption instead of title
%escapeinside={\%*}{*)},         % if you want to add a comment within your code
%xleftmargin=1cm,
%xrightmargin=1cm,
%morekeywords={*,...}            % if you want to add more keywords to the set
%}
%

%\newminted{perl}{linenos, bgcolor=light-gray, fontsize=\scriptsize}
%\newminted{cpp}{bgcolor=light-gray, fontsize=\scriptsize}
%\newminted{tcl}{bgcolor=light-gray, fontsize=\scriptsize}
%\newminted{sh}{bgcolor=light-gray, fontsize=\scriptsize}
%\newminted{basemake}{bgcolor=light-gray, fontsize=\scriptsize}

%%%%%%%%%%%%%%%%%%%%%%%%%%%%%%%%%}}}%
% fancy nomenclautur:
%%%%%%%%%%%%%%%%%%%%%%%%%%%%%%%%%{{{%
%\setlength{\nomlabelwidth}{.20\hsize}
%\renewcommand{\nomlabel}[1]{#1 \dotfill}

%<*sample05>
\def\@@@nomenclature[#1]#2#3{%
 \def\@tempa{#2}\def\@tempb{#3}%
 \protected@write\@nomenclaturefile{}%
  {\string\nomenclatureentry{#1\nom@verb\@tempa @[{\nom@verb\@tempa}]%
    |nompageref{\begingroup\nom@verb\@tempb\protect\nomeqref{\theequation}}}%
    {\thepage}}%
 \endgroup
 \@esphack}
%\def\nompageref#1#2{%
%  \if@printpageref\pagedeclaration{#2}\else\null\fi
%  \linebreak#1\nomentryend\endgroup}
\def\pagedeclaration#1{\dotfill\nobreakspace ~#1}
%\def\nomentryend{.}
\def\nomlabel#1{\textbf{#1}\hfil}
\makeatletter 
\renewcommand*\dotfill{\leavevmode% 
  \leaders\hbox{$\m@th 
  \mkern \@dotsep mu\hbox{.}\mkern \@dotsep 
  mu$}\hfill\kern\z@} 
\makeatother
%%%%%%%%%%%%%%%%%%%%%%%%%%%%%%%%%}}}%
% Abbr Commands!
%%%%%%%%%%%%%%%%%%%%%%%%%%%%%%%%%{{{%
\newcommand{\abbr}[2]{\textit{#2} (#1)\nomenclature{#1}{#2 \nomrefpage}}
\newcommand{\shortabbr}[2]{\nomenclature{#1}{#2 \nomrefpage}}
\newcommand{\revabbr}[2]{#1 (\textit{#2})\nomenclature{#1}{#2 \nomrefpage}}


%%%%%%%%%%%%%%%%%%%%%%%%%%%%%%%%%}}}%
% Titlepage                         %
%%%%%%%%%%%%%%%%%%%%%%%%%%%%%%%%%{{{%
%%%%%%%%%%%%%%%%%%%%%%%%%%%%%%%%%%%%%%%%%%%%%%%%%%%%%%%%%%%%%%%%%%%%%%%%%%%%%%%%%%%%%%%%%%%%%%%%%%%
\begin{titlepage}
\setcounter{page}{1}
\begin{center}
\includegraphics*[scale=2.5]{img/TUKL_LOGO.pdf}\\[3ex]

\textsc{\Large University of Kaiserslautern}\\[1.5ex]
Department of Electrical Engineering and Information Technology\\[1.5ex]
Microelectronic Systems Design Research Group \\[3ex]

\vfill
\vfill

\textsc{\Huge Bachelor Thesis}\\[6ex]
\centerline{\Large Design and Implementation of a Blockchain-Based Smart Outlet Concept}
\vspace{20pt}
\centerline{\Large Entwurf und Implementierung eines Blockchain-basierten Smart Outlets Konzept}

\vfill
\vfill

 \begin{tabular}{rl}\hline\\
 Presented:                & \quad \today \\[1.5ex]
 Author:                   & \quad Daniel Gretzke (392488) \\[1.5ex]
 Research Group Chief:     & \quad Prof.\,Dr.-Ing.\,~N.~Wehn\\[1.5ex]
 Tutor:                    & \quad M.Sc. Frederik Lauer\\[1.5ex]\\\hline
 \end{tabular}
\end{center}

    \clearpage
    \pagestyle{empty}
    \begin{flushleft}
    \section*{Statement}
    \vspace{10mm}
    I declare that this thesis was written solely by myself and exclusively with
    help of the cited resources.

    \vspace{12pt}
    Kaiserslautern, \today \\
%    Kaiserslautern, XXst XXXXXX 2010 \\
    \vspace{20mm}
    Daniel Gretzke
    \end{flushleft}

\end{titlepage}


\newpage

%\clearpage{\pagestyle{empty}\cleardoublepage}
\newpage
% !TEX root = ../diss.tex

\begin{abstract}
\thispagestyle{empty}

\section*{Abstract}

\textcolor{red}{Abstract English here.}

\end{abstract}

%\clearpage{\pagestyle{empty}\cleardoublepage}
\newpage
% !TEX root = ../diss.tex

\begin{abstract}
\thispagestyle{empty}

\section*{Zusammenfassung}

\textcolor{red}{Abstract Deutsch here.}

\end{abstract}

\newpage
%%%%%%%%%%%%%%%%%%%%%%%%%%%%%%%%%}}}%
% Table of Contents                 %
%%%%%%%%%%%%%%%%%%%%%%%%%%%%%%%%%{{{%
\tableofcontents
\newpage
\setcounter{page}{1}
\newpage
%%%%%%%%%%%%%%%%%%%%%%%%%%%%%%%%%}}}%
% Chapters                          %
%%%%%%%%%%%%%%%%%%%%%%%%%%%%%%%%%{{{%
%\onehalfspacing % Stelle 1.5er Abstand ein
%\setstretch{1.1} 
\section{Introduction}
\begin{quote}
  Whereas most technologies tend to automate workers on the periphery doing menial tasks, blockchains automate away the center. Instead of putting the taxi driver out of a job, blockchain puts Uber out of a job and lets the taxi drivers work with the customer directly.
  \\
  {\textit{— Vitalik Buterin, co-founder of Ethereum}}
\end{quote}

A cryptocurrency based on a blockchain was first implemented in 2009 by Satoshi Nakamoto (pseudonym) and was called Bitcoin.
Since then, it has steadily gained importance every year.
Meanwhile, thousands of cryptocurrencies and tokens were built on this technology.
The hype in the year 2017 called the attention of many companies to blockchain and even last year, when the value of cryptocurrencies fell as far as 95\%, the interest in this field did not drop.
\\\\
Compared to traditional payment methods like Visa, Banks and PayPal, cryptocurrencies are built decentralized, meaning that there is no central organization that controls transactions, the issuance of new money, et cetera.
The validity of the blockchain \abbr{peer to peer}{P2P} network is secured through cryptographic protocols.
This brings several benefits.
Traditional payment methods usually go with high transactions costs, most commonly in the amount of a few percent.
On the contrary, the cost of a single transaction on a blockchain averages out at just a few cents\cite{ethereum-fee}.
Some cryptocurrencies even work without any fees.
\\\\
Because of this, they are suited for micro transactions really well.
There are some disadvantages, though.
The blockchain technology is still at an early stage and really immature.
Compared to traditional electronic payments, it only manages to achieve very few \abbr{transactions per second}{TPS} and has long transaction times.
E.g., Bitcoin manages 4-5 TPS\cite{bitcoinTPS} as opposed to Visa, which manages to process almost 4,000 TPS on average\cite{visa}.
\\\\
As stated in the quote above, the key strength of blockchain and cryptocurrencies is the decentralization aspect.
For many, it will reshape various markets we know today, potentially revolutionize the financial industry and even disrupt monopolies in the future.
\\\\
Another trend regarding the future are electric cars.
It's expected that in a few years most cars on the road and almost all cars sold will be electric.
Often these need to be charged overnight.
Unfortunately, most city residents are familiar with the problem that they rarely park in front of their own house, let alone own a garage.
It's foreseeable that recharging a car might bring difficulties.
\\\\ 
This bachelor thesis is devoted to this problem.
It examines whether a smart electrical socket, which is placed outside the house by a homeowner, can be used to efficiently sell electricity and which payment method is suited best for this task.
Based on an initial concept, a prototype is to be developed that implements the previously worked out features.
It will serve as an example on how to implement \abbr{machine to machine}{M2M} payments on a microcontroller level.
\newpage
\clearpage
%\input{2.usw.usw}
%\newpage
%\clearpage
%%%%%%%%%%%%%%%%%%%%%%%%%%%%%%%%%}}}%
% Appendix                          %
%%%%%%%%%%%%%%%%%%%%%%%%%%%%%%%%%{{{%
\newpage
\clearpage

%clear headers
\fancyhead{}
\fancyfoot{}
\fancyfoot[CO, CE] {\thepage}

%\section{Appendix}

%%%%%%%%%%%%%%%% >INSERT YOUR APPENDIX HERE>
\section{Appendix}
\label{sec:appendix}


%\begin{listing}[H]
%%\begin{perlcode}
%\begin{minted}[linenos, bgcolor=light-gray, fontsize=\scriptsize]{perl}
%#!/usr/bin/perl
%
%use strict;
%use warnings;
%use POSIX;
%
%# University of Kaiserslautern 2014
%# Matthias Jung 
%# Christian Weis
%# Peter Ehses
%# programm call: perl error_detecta.pl input output
%
%my $input   = $ARGV[0];
%my $id      = "022804e800";
%my $pattern = hex("FFFFFFFF");  # data pattern can be changed to AAAAAAAA or 55555555
%my $errors  = 0;
%my $i = 1;
%my $addr = 0;                   # DRAM address in hex
%my $addrb = 0;                  # 23 bit in binary
%my $addroffset = 536870912;     # offset for the address in hex 0x20000000
%my $bank =0;                    # 2 bits for the bank
%my $bankshift =21;              # shiftoperator
%my $bankand = 6291456;          # 2^22+2^21
%my $row =0;                     # 12 bits for the row
%my $rowshift =9;                # shiftoperator
%my $rowand = 2096640;           # 2^20+...+2^9
%my $column =0;                  # 9 bits for the column
%my $columnand = 511;            # 2^8+...+2^0
%
%open(IFH, $input);
%open(out_file, ">errorout_$ARGV[1]");
%
%printf out_file ("The following table shows the addresses ");
%printf out_file ("from the errors in the wideIO SDRAM.\n");
%printf out_file ("Addresses in binary \t\t\t Addresses in hexadecimal\n");
%printf out_file ("bank \t row \t\t column \t SDRAM address \t data value\n");
%
%while(<IFH>)
%{
%  unless($_ =~ /\[.*\]/ || $_ =~ /$id/)
%  {
%    my $value = $_;
%    chop($value);
%    $value = substr( $value , 2);
%    my $result = sprintf("%0b", (hex($value) ^ $pattern));
%    for(my $j = 0; $j < length($result); $j ++)
%    {
%      if(substr( $result, $j , 1 ) eq "1")
%      {
%        $errors++;
%        $addrb = $i-((ceil($i/11))*3);
%        $addr = $addrb + $addroffset;
%        $bank = ($addrb & $bankand) >> $bankshift;
%        $row = ($addrb & $rowand) >> $rowshift;
%        $column = ($addrb & $columnand);
%        printf out_file ("%02b\t %012b\t %09b\t %#8x\t $value\n", $bank, $row, $column, $addr);
%      }
%    }
%  }
%  $i++;
%}
%
%close(out_file);
%print "Errors = ".$errors."\n";
%close(IFH);
%%\end{perlcode}
%\end{minted}
%\caption{Perl script to find errors for data pattern F, A or 5}
%\label{lis:5af}
%\end{listing}

\pagebreak
Pagebreak and linebreak has to be done manually with pygmentize, this feature is
not yet implemented. Open the appendix.tex file and see the source code
afterwards how the pagebreak is done. For that the appendix.tex has to be
written with pagebreaks, so that the layout of the pages is done manually.

Linebreaks are easier to do, just check that the lines are in the box of the pdf
file, otherwise make a linebreak yourself.
\pagebreak

%%\begin{listing}[H]
%\begin{minted}[linenos, bgcolor=light-gray, fontsize=\scriptsize]{perl}
%#!/usr/bin/perl
%
%use strict;
%use warnings;
%use POSIX;
%use Chart::Gnuplot;
%
%# University of Kaiserslautern 2014
%# Matthias Jung 
%# Christian Weis
%# Peter Ehses
%# call programm: perl plotreffff_0xf.pl dfile1 dfile2 dfile3
%
%my $i = 0;
%my $line = 3;
%my $addr = 0;           # DRAM address in hex
%my $bankb;              # 2 bits for the bank
%my $rowb;               # 12 bits for the row
%my $columnb;            # 9 bits for the column
%my $bank;               # banknumber in decimal
%my $row;                # rownumber in decimal
%my $column;             # columnnumber in decimal
%my $value;
%my @ytics = [0, 25,50,75,100,125,150,175,200,225,250,275,300,325,350,375,400,425,450,475,500];
%my @xtics = [0,250,500,750,1000,1250,1500,1750,2000,2250,2500,2750,3000,3250,3500,3750,4000];
%
%# set terminal to svg format
%my $terminal = 'svg mouse jsdir '.'"http://gnuplot.sourceforge.net/demo_svg"';
%sub bin2dec {return unpack("N", pack("B32", substr("0" x 32 . shift, -33)));}
%
%my @row_array;
%my @column_array;
%
%foreach my $argnum (0 .. $#ARGV)
%{
%  open(IFH, $ARGV[$argnum]);
%  $i = 0;
%  while(<IFH>)
%  { 
%    chomp;
%    $i++;
%    if ($i > $line)
%    {
%      ($bankb, $rowb, $columnb, $addr, $value) = split("\t");	
%      $bank = (bin2dec($bankb));
%      $row = (bin2dec($rowb));
%      $column = (bin2dec($columnb));
%      if ($argnum == 0)
%      {
%        push(@{$row_array[$bank]}, $row);
%        push(@{$column_array[$bank]}, $column);
%      }
%      if ($argnum == 1)
%      {
%        push(@{$row_array[$bank+4]}, $row);
%        push(@{$column_array[$bank+4]}, $column);
%      }
%      if ($argnum == 2)
%      {
%        push(@{$row_array[$bank+8]}, $row);
%        push(@{$column_array[$bank+8]}, $column);
%      }
%    }
%  }
%  close(IFH);
%}
%for (my $count = 1; $count < 5; $count++)
%{
%  $bank = $count -1;
%\end{minted}

%%here is a pagebreak, and the next line of the code is starting with 70, has to be specified with minted like below.
%\begin{listing}[H]
%\begin{minted}[linenos, bgcolor=light-gray, fontsize=\scriptsize, firstnumber=70]{perl}
%  my $plot1 = Chart::Gnuplot->new(
%  terminal => $terminal, output => "plot_ref202ms_0xf_b_$count.svg",
%  title => "Errors channel 3 of SDRAM, bank $count, data pattern 0xffffffff and refresh 202 ms",
%  imagesize => '1024, 768', xlabel => "row address", ylabel => "column address", yrange=>[0, 511],
%  xrange=>[0, 4095], ytics => {labels => @ytics}, xtics => {labels => @xtics},
%  legend => {position => "outside center bottom", order =>"horizontal reverse",
%  border => "on", align => "left"}
%  );
%  my $dataSet1 = Chart::Gnuplot::DataSet->new(
%  xdata => \@{$row_array[$bank]}, ydata => \@{$column_array[$bank]},
%  color => "blue", pointtype => 6, pointsize => 1.75, width => 2,
%  title => "95 degree C"
%  );
%  my $dataSet2 = Chart::Gnuplot::DataSet->new(
%  xdata => \@{$row_array[$bank+4]}, ydata => \@{$column_array[$bank+4]},
%  color => "red", pointtype => 8, pointsize => 1.25, width => 2,
%  title => "100 degree C"
%  );
%  my $dataSet3 = Chart::Gnuplot::DataSet->new(
%  xdata => \@{$row_array[$bank+8]}, ydata => \@{$column_array[$bank+8]},
%  color => "dark-green", pointtype => 10, pointsize => 1.25, width => 2,
%  title => "105 degree C"
%  );
%  
%  if (@{$row_array[$bank]}){$plot1->plot2d($dataSet1, $dataSet2, $dataSet3);}
%  if (!@{$row_array[$bank]}){$plot1->plot2d($dataSet2, $dataSet3);}
%  if (!@{$row_array[$bank]} && !@{$row_array[$bank+4]}){$plot1->plot2d($dataSet3);}
%}
%\end{minted}
%\caption{Perl script for scatter plot of different refresh periods}
%\label{lis:plotref}
%\end{listing}


\newpage
\clearpage
%\addcontentsline{toc}{section}{Appendix}
%List of Figures
\vspace{-20pt}
\begingroup
    \addcontentsline{toc}{subsection}{List of Figures}
    \setlength{\cftparskip}{10pt}
    \listoffigures 
\endgroup
\newpage
\clearpage
%List of Tables
\begingroup
    \addcontentsline{toc}{subsection}{List of Tables}
    \setlength{\cftparskip}{10pt}
    \listoftables
\endgroup
\newpage
\clearpage
%List of Listings
%\renewcommand{\lstlistlistingname}{Verzeichnis der Quellcodes}
\begingroup
    \addcontentsline{toc}{subsection}{List of Listings}
      \setlength{\itemsep}{20pt}
  \setlength{\parskip}{10pt}
    \renewcommand{\listlistingname}{List of Listings}
    \listoflistings 
\endgroup
\newpage
\clearpage
%List of Abbreviations
\begingroup
    \addcontentsline{toc}{subsection}{List of Abbreviations}
    \renewcommand{\nomname}{List of Abbreviations}
    \renewcommand{\nompreamble}{\vspace{10pt}}
    %\setlength{\nomitemsep}{8pt}
    \printnomenclature[2cm]
\endgroup
\newpage
\clearpage
%Literatur:
\addcontentsline{toc}{subsection}{References}
\bibliographystyle{unsrt}
\bibliography{doc}
%%%%%%%%%%%%%%%%%%%%%%%%%%%%%%%%%}}}%
\end{document}
